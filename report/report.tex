\documentclass{article}

\usepackage[english]{babel}

\usepackage[letterpaper,top=2cm,bottom=2cm,left=3cm,right=3cm,marginparwidth=1.75cm]{geometry}

\usepackage{amsmath}
\usepackage{graphicx}
\usepackage[colorlinks=true, allcolors=blue]{hyperref}
\usepackage{natbib}
\bibliographystyle{alpha}
\usepackage{caption}
\usepackage{float}

\title{Aprendizado de Máquina \\ Trabalho Prático 2}
\author{Luís Felipe Ramos Ferreira \\ 2019022553 \\
    \href{mailto:lframos_ferreira@outlook.com}{\texttt{lframos\_ferreira@outlook.com}}}

\begin{document}
\maketitle

\section{Introdução}

O Trabalho Prático 2 da disciplina de Aprendizado de Máquina teve como objetivo o desenvolvimento de um algoritmo de \textit{boosting}
para classificação binária. Em particular, o algoritmo a ser desenvolvido é o \href{https://en.wikipedia.org/wiki/AdaBoost}{\textit{Adaboost}} e a base de dados a ser
utilizada nos testes é o conjunto \href{https://archive.ics.uci.edu/ml/datasets/Tic-Tac-Toe+Endgame}{\textit{Tic-Tac-Toe}}. Além disso, os modelos criados deveriam
ser analisados por meio da metodologia de validação cruzada com 5 partições para avaliação do modelo.

\section{Implementação}

A linguagem escolhida para o desenvolvimento do trabalho foi \href{https://www.python.org/}{\texttt{Python}} (versão 3.10), devida a sua grande variedade de bibliotecas úteis para ciência de dados e aprendizado de máquina.
 A modelagem do algoritmo \textit{AdaBoost} foi feita com o uso dE bibliotecas de análise numérica como \href{https://numpy.org/}{\texttt{NumPy}} e \href{https://pandas.pydata.org/}{\texttt{Pandas}}, 
 uma vez que se tratam de ferramentas extremamente completas que facilitaram o desenvolvimento do algoritmo.

Para organizar o ambiente de desenvolvimento, que englobava vários pacotes diferentes, foi utilizado o gerenciador de pacotes \href{https://www.anaconda.com/}{\texttt{Anaconda}}, o que facilitou o trabalho
com os pacotes de ciência de dados citados. O projeto final foi salvo em um \href{https://github.com/lframosferreira/boosting-process}{\texttt{repositório}} no GitHub para fácil versionamento e organização de código.

\section{Experimentos}

Os experimento

\section{Análise dos resultados de teste}

De maneira ger

\section{Convergência do erro empírico}

Durante o treinamento das redes neuronais propostas, o histórico do erro empírico pode ser armazenado para análise de sua convergência, considerando cada configuração de rede proposta. Para fins de simplificação, serão mostrados
aqui

\section{Conclusão}

Em suma, após as análises e discussões apresentadas neste relatório, fica claro que os parâmetros da rede neuronal, como o número de neurônios na camada oculta,


\end{document}